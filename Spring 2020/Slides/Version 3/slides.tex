\documentclass{beamer}

\usepackage[utf8]{inputenc}
\usepackage{amsmath}
\usepackage{mathtools}
\usepackage{hyperref}
\usepackage{enumitem}



%Information to be included in the title page:
\title{Presentation for the Quantum Seminar}
\author{Jos\'e Manuel Rodr\'iguez Caballero}
\institute{University of Tartu}
\date{Spring 2020}



\begin{document}

\frame{\titlepage}

\begin{frame}{Subject}
My presentation is about the paper\footnote{I will not repeat the notation from the paper in this presentation. If someone needs a clarification, please, either ask me during the presentation or read Bouman-Fehr paper.}:
$\newline$

\begin{quote}
Bouman, Niek J., and Serge Fehr. "Sampling in a quantum population, and applications." Annual Cryptology Conference. Springer, Berlin, Heidelberg, 2010.
URL = \url{https://arxiv.org/pdf/0907.4246.pdf}
\end{quote}
\end{frame}


\begin{frame}{Outline}
The main contributions of Bouman-Fehr paper are the following.

\begin{enumerate}[label=(\Roman*)]
\item Introduction of a theory of \emph{sampling and estimate strategies} for classical and quantum populations.
\item A new proof of the security of the protocol for quantum key distribution BB84 (and the entanglement-based version of it).
\item A new proof of the security of the protocol Quantum Oblivious Transfer\footnote{We consider that (i) and (ii) are enough in order to understand the technique developed Bouman-Fehr paper. So, we will omit (iii) in this presentation because of time constrains.} (QOT).
\end{enumerate}
\end{frame}

\begin{frame}{Brief History}
\begin{enumerate}[label=(\roman*)]
\item The protocol BB84, developed by Charles Bennett and Gilles Brassard\footnote{C. H. Bennett and G. Brassard, “Quantum cryptography: Public-key distribution and coin tossing,” in Proceedings of IEEE International Conference on Computers,
Systems and Signal Processing, Bangalore, India, 1984,
(IEEE Press, 1984), pp. 175–179} in 1984, was the first quantum key distribution protocol.
\item An entanglement-based version of BB84 was proposed by Artur K. Ekert\footnote{Artur K. Ekert. Quantum cryptography based on Bell's theorem. Physical Review Letter,
67(6):661–663, August 1991.} in 1991. The security of this version of BB84 implies the security of the original protocol.
\item The first security proof of BB84 was published by Dominic Mayers\footnote{Mayers, D. 1996. Quantum key distribution and string oblivious transfer in noisy channels.
Advances in Cryptology--Proceedings of Crypto '96 (Aug.). Springer-Verlag, New York, pp. 343–357} in 1996.
\end{enumerate}
\end{frame}

\begin{frame}{Description}
Let $n \geq 2$ and $1 \leq k \leq \frac{n}{2}$ be the integer parameters of the following protocol. The entanglement-based BB84 protocol can be divided into the following steps\footnote{The explanation of each step will be developed in the next slides}. $\newline$

\begin{enumerate}[label=(\roman*)]
\item Qubit distribution.
\item Error estimation.
\item Error correction.
\item Key distillation.
\end{enumerate}
\end{frame}

\begin{frame}{Qubit distribution}

\begin{enumerate}[label=(\roman*)]
\item Alice prepare $n$ EPR pairs $\frac{1}{\sqrt{2}}\left( |00\rangle + |11\rangle \right)$.
\item Alice sends one qubit  for each pair to Bob.
\item Bob confirms the receipt of the qubits.
\item Alice picks random $\theta\in\{0,1\}^n$ and send it to Bob.
\item Alice and Bob measure their respective qubits in basis $\theta$ ($0$ for computational, $1$ for Hadamard) and the results of the measurements are registered in $x$ and $y$ respectively.
\end{enumerate}
\end{frame}

\begin{frame}{Error estimation}
\begin{enumerate}[label=(\roman*)]
\item Alice chooses a random subset $s \subset [n]$ of size $k$ and send it to Bob.
\item Alice and Bob exchange $x_s$ and $y_s$.
\item Alice and Bob both compute $\omega\left( x_s \oplus y_s\right)$.
\end{enumerate}
\end{frame}

\begin{frame}{Error correction}
\begin{enumerate}[label=(\roman*)]
\item Alice send the syndrome $\textbf{syn}$ of $x_{\overline{s}}$ to Bob with respect to a suitable linear error correcting code. Let $m$ be the bit-size of $\textbf{syn}$.
\item Bob uses $\textbf{syn}$ to correct the errors in $y_{\overline{s}}$ and obtains  $\hat{x}_{\overline{s}}$.
\end{enumerate}
\end{frame}

\begin{frame}{Key distillation}
\begin{enumerate}[label=(\roman*)]
\item Alice chooses a random seed $r$ for a universal hash function $g$ with range $\{0,1\}^{\ell}$, where $\ell < \left(1 - h(\beta)\right) n - k - m$ (or $\ell = 0$ if the right-hand side is not positive).
\item Alice sends $r$ to Bob.
\item Alice and Bob compute their keys $\mathbf{k} := g(r, x_{\overline{s}})$ and $\mathbf{\hat{k}} := g(r, \hat{x}_{\overline{s}})$.
\end{enumerate}
\end{frame}

\begin{frame}{Security claim (statement)}

Consider an execution of the entanglement-based BB84 in the presence of an adversary Eve. Let $\mathbf{K}$ be the key obtained by Alice, and let $E$ be Eve's quantum system at the end of the protocol. Let $\mathbf{\tilde{K}}$ be chosen uniformly at random of the same bit-length as $\mathbf{K}$. Then, for any $0 < \delta \leq \frac{1}{2}-\beta$, the inequality
\begin{eqnarray*}
\Delta\left( \rho_{\mathbf{K}E}, \rho_{\mathbf{\tilde{K}}E} \right) &\leq& \frac{1}{2}  \exp\left[ -\frac{\ln 2}{2} \bigg( (1 - h(\beta+\delta))n - k - m - \ell \bigg) \right] \\
& & + 2\exp\left( -\frac{\delta^2 k}{6} \right)
\end{eqnarray*}
holds.

\end{frame}

\begin{frame}{Security claim (application)}
Let  $\varepsilon > 0$. The security claim can be used in order compute a possible value for $\ell$ such that $\Delta\left( \rho_{\mathbf{K}E}, \rho_{\mathbf{\tilde{K}}E} \right) \leq \varepsilon$.
\end{frame}

\begin{frame}{Security proof (sketch)}
There is a quantum state $\rho_{\beta,\delta}$ satisfying the following conditions.
\begin{enumerate}[label=(\roman*)]
\item Quantum error: $$\Delta\left( \rho_{\mathbf{K}E}, \rho_{\beta,\delta} \right) \leq 2\exp\left( -\frac{\delta^2 k}{6} \right).$$
\item Privacy amplification: $$\Delta\left(\rho_{\beta,\delta}, \rho_{\mathbf{\tilde{K}}E} \right) \leq \frac{1}{2} \exp\left[ -\frac{\ln 2}{2} \bigg( (1 - h(\beta+\delta))n - k - m - \ell \bigg) \right].$$
\end{enumerate}

Applying triangular inequality we get the desired result.
\end{frame}

\begin{frame}{Motivation}
In this presentation, the motivation for introducing the theory of sampling and estimation strategies for quantum and classical populations is to guarantee the existence of the quantum state $\rho_{\beta,\delta}$ satisfying the conditions of the previous slide.
\end{frame}

\begin{frame}{Main definition}
Let $\mathcal{I}$ be a finite set of indices, $\mathcal{S}$ be a finite set of seeds and $\mathcal{A}$ be a finite alphabet. Define $\mathcal{T} := 2^{\mathcal{I}}$. A \emph{sampling and estimation strategy} (a \emph{strategy} for short) is given\footnote{Our definition is slightly different of the definition given in Bouman-Fehr paper, but equivalent to it.} by $\Psi := \left(\mathcal{A}, \mathcal{I}, \mathcal{S}, P_{TS}, f \right)$, where $P_{TS}$ is a probability distribution over $\mathcal{T} \times \mathcal{S}$ and $f$ is a real-valued function over 
$$
\text{Dom}_f := \bigcup_{(t, s)\in\mathcal{T}\times\mathcal{S}} \left\{ (t, q, s): \quad q \in \mathcal{A}^t \right\}.
$$
\end{frame}

\begin{frame}{Main example}
\textbf{Strategy $\Psi_{n,k}$:} Pairwise one-out-of-two sampling, using only part of the sample.
$\newline$

Consider the integer parameters $n \geq 2$ and $1 \leq k \leq \frac{n}{2}$. Let $\mathcal{A} := \{0,1\}$, $\mathcal{I} := [n] \times \{0,1\}$ and $\mathcal{S} := \mathcal{T}$. The probability distribution $P_{TS}$ is given by 
$$
P_{TS}(t,s) = \frac{1}{2^n \left( \!\! \begin{array}{c} n \\ k \end{array} \!\! \right)}
$$
if for some $(j_1,...,j_n)\in \{0,1\}^n$ we have $t = \{(\ell, j_{\ell}): \quad 1 \leq \ell \leq n\}$, $|s| = k$ and $s \subset t$. Otherwise, $P_{TS}(t,s) := 0$. Furthermore, $f(t, q, s) := \omega(q_s)$.
\end{frame}

\begin{frame}{Existence of $\rho_{\beta,\delta}$ and quantum error (sketch)}
\begin{enumerate}[label=(\roman*)]
\item The error estimation phase in the protocol is interpreted as strategy $\Psi_{n,k}$. 
\item The classical error of the strategy is bounded above by $4\exp\left( -\frac{\delta^2 k}{3} \right)$ using well-known techniques from probability theory (Hoeffding's inequality).
\item The quantum error of the strategy is bounded above by the square root of the classical error of the strategy, i.e., $2\exp\left( -\frac{\delta^2 k}{6} \right)$.
\item According to the definition of the quantum error, there exists a state $\rho_{\beta,\delta}$, which is a superposition of states having relative Hamming weight\footnote{The basis from which the Hamming weight is taken are the Hadamard basis for Alice and the computational
basis for Bob.} $\delta$-close to $\beta$, for which $\Delta\left( \rho_{\mathbf{K}E}, \rho_{\beta,\delta} \right)$ is bounded above by the quantum error, \emph{a fortiori} by $2\exp\left( -\frac{\delta^2 k}{6} \right)$.
\end{enumerate}
\end{frame}

\begin{frame}{Privacy amplification (sketch)}
\begin{enumerate}[label=(\roman*)]
\item The fact that $\rho_{\beta,\delta}$ is a superposition of states having relative Hamming weight $\delta$-close to $\beta$ implies the inequality
$$
H_{\min}(W|\Theta \, Z \, S \, E_0) \geq \left(1 - h(\beta + \delta) \right) n.
$$
\item Applying the chain rule to the inequality above, we get
$$
H_{\min}(X_{\overline{S}}|\Theta \, Z \, X_S \, \textbf{SYN} \, E_0) \geq \left(1 - h(\beta + \delta) \right) n - k - m.
$$

\item In virtue of the privacy amplification inequality, $\Delta\left(\rho_{\beta,\delta}, \rho_{\mathbf{\tilde{K}}E} \right) \leq \frac{1}{2}\cdot 2^{\left(H_{\min}(X_{\overline{S}}|\Theta \, Z \, X_S \, \textbf{SYN} \, E_0) - \ell \right)/2}$.
\end{enumerate}

From (ii) and (iii) we conclude the desired inequality.
\end{frame}

\begin{frame}
\begin{center}
\Large{\textbf{End of my presentation} }
\end{center}
\end{frame}

\end{document}