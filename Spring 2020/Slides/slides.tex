\documentclass{beamer}

\usepackage[utf8]{inputenc}
\usepackage{amsmath}
\usepackage{mathtools}


%Information to be included in the title page:
\title{Presentation for the Quantum Seminar}
\author{Jos\'e Manuel Rodr\'iguez Caballero}
\institute{University of Tartu}
\date{Spring 2020}



\begin{document}

\frame{\titlepage}

\begin{frame}{Subject}
My presentation is about the paper\footnote{I will not repeat the notation from the paper in this presentation. If someone needs a clarification, either read the paper or ask me during the presentation.}:
$\newline$

\begin{quote}
Bouman, Niek J., and Serge Fehr. "Sampling in a quantum population, and applications." Annual Cryptology Conference. Springer, Berlin, Heidelberg, 2010.
\end{quote}

My presentation may differ in some points with respect of the presentation given by the authors.

\end{frame}

\begin{frame}{Contributions of the paper}
\begin{flushleft}

\textbf{Contribution 1.} The authors introduce a framework in for \textit{sampling quantum population}.
$\newline$

\textbf{Contribution 2.} This framework is used in a new proof of the security of the \emph{quantum key distribution protocol BB84} (entanglement-based version).
$\newline$

\textbf{Contribution 3.} This framework is used in a new proof of the security of the \emph{quantum oblivious-transfer from bit-commitment}.

\end{flushleft}
\end{frame}

\begin{frame}
\begin{center}
\Large{\textbf{Presentation of Contribution 1.} }\normalsize
$\newline$
\end{center}

\begin{flushleft}
The authors introduce a framework in for \textit{sampling quantum population}.
\end{flushleft}
\end{frame}

\begin{frame}{Classical sampling strategy}
Let $n$ be a positive integer, $\mathcal{A}$ be a finite alphabet and $\mathcal{S}$ be a finite set of seeds\footnote{Both ``alphabet" and ``seeds" are informal labels notions here in order to show the motivation for introducing these sets.}. A \emph{classical sampling strategy} is a triplet $\left( P_T, P_S, f \right)$, where $P_T$ is a probability distribution over $\mathcal{T} := 2^{[n]}$, $P_S$ is a probability distribution over $\mathcal{S}$ and $f$ is a real-valued function defined on the finite set
$$
\left\{ (t, q, s)\in \mathcal{T}\times \mathcal{A}^{\ast} \times \mathcal{S}: \quad |t| = |q| \right\}.
$$
\end{frame}


\begin{frame}{Set of classical $\delta$-close states} 
Let $\delta$ be a positive real number. We define the  \emph{set of classical $\delta$-close states} as
$$
B_{t,s}^{\delta} := \left\{ q\in\mathcal{A}^n: \quad \left| \omega\left(q_{\overline{t}}\right) - f\left(t, q_t, s\right) \right| <  \delta \right\},
$$
where $\omega(q)$ is the Hamming density of $q$, i.e., the Hamming weight of $q$ divided by the length of $q$.

\end{frame}

\begin{frame}{Random set of classical $\delta$-close states} 
Let $T$ and $S$ be random variables associated to the probability distributions $P_T$ and $P_S$ respectively. Notice that the pair $(T, S)$ is a random variable. Furthermore, the evaluation of $(t,s) \mapsto B_{t,s}^{\delta}$ at $(T,S)$ determines a random variable, denoted $B_{T,S}^{\delta}$ and associated to the probability distribution
$$
\textrm{Pr}\left[ B_{T,S}^{\delta} \in \Gamma \right] := \textrm{Pr}\left[ (T, S) \in \left\{(t,s)\in\mathcal{T}\times\mathcal{S} :\quad B_{t,s}^{\delta} \in \Gamma\right\} \right]
$$
where $\Gamma\subset 2^{\mathcal{A}^n}$. The call $B_{T,S}^{\delta}$ the \emph{random set of classical $\delta$-close states}.
\end{frame}


\begin{frame}{Classical $\delta$-error} 

The \emph{classical $\delta$-error} is defined as\footnote{Notice that the expression inside the bracket is equivalent to $q \not\in B_{T,S}^{\delta}$.}
$$
\varepsilon_c^{\delta} :=\max_{q\in\mathcal{A}^n} \textrm{Pr}\left[ B_{T,S}^{\delta} \in \left\{X\in 2^{\mathcal{A}^n}: \quad q \not\in X \right\}\right].
$$

\end{frame}


\begin{frame}
\begin{center}
\Large{\textbf{Presentation of Contribution 2.} }\normalsize
$\newline$
\end{center}

\begin{flushleft}
This framework is used in a new proof of the security of the \emph{quantum key distribution protocol BB84} (entanglement-based version).
\end{flushleft}
\end{frame}

\begin{frame}
\begin{center}
\Large{\textbf{Presentation of Contribution 3.} }\normalsize
$\newline$
\end{center}

\begin{flushleft}
This framework is used in a new proof of the security of the \emph{quantum oblivious-transfer from bit-commitment}.
\end{flushleft}
\end{frame}

\begin{frame}
\begin{center}
\Large{\textbf{End of my presentation} }
\end{center}
\end{frame}

\end{document}