\documentclass{beamer}

\usepackage[utf8]{inputenc}
\usepackage{amsmath}
\usepackage{mathtools}


%Information to be included in the title page:
\title{Presentation for the Quantum Seminar}
\author{Jos\'e Manuel Rodr\'iguez Caballero}
\institute{University of Tartu}
\date{Spring 2020}



\begin{document}

\frame{\titlepage}

\begin{frame}{Subject}
My presentation is about the paper\footnote{I will not repeat the notation from the paper in this presentation. If someone needs a clarification, either read the paper or ask me during the presentation.}:
$\newline$

\begin{quote}
Bouman, Niek J., and Serge Fehr. "Sampling in a quantum population, and applications." Annual Cryptology Conference. Springer, Berlin, Heidelberg, 2010.
\end{quote}

My presentation may differ in some points with respect of the presentation given by the authors.

\end{frame}

\begin{frame}{Contributions of the paper}
\begin{flushleft}

\textbf{Contribution 1.} The authors introduce a framework in for \textit{sampling quantum population}.
$\newline$

\textbf{Contribution 2.} This framework is used in a new proof of the security of the \emph{quantum key distribution protocol BB84} (entanglement-based version).
$\newline$

\textbf{Contribution 3.} This framework is used in a new proof of the security of the \emph{quantum oblivious-transfer from bit-commitment}.

\end{flushleft}
\end{frame}

\begin{frame}
\begin{center}
\Large{\textbf{Presentation of Contribution 1.} }\normalsize
$\newline$
\end{center}

\begin{flushleft}
The authors introduce a framework in for \textit{sampling quantum population}.
\end{flushleft}
\end{frame}

\begin{frame}{Overview of Contribution 1}

The authors define a \emph{sampling and estimation strategy} (\emph{sampling strategy} for short). This strategy can be used in order to study both classical and quantum populations. The ``quantum error" is bounded by the square root of the ``classical error", thus reducing the quantum information-theoretical problem to a well-studied probabilistic problem.
\end{frame}

\begin{frame}{Sampling strategy}
Let $n$ be a positive integer, $\mathcal{A}$ be a finite alphabet and $\mathcal{S}$ be a finite set of seeds. Assume that $|\mathcal{A}| \geq 2$ and $|\mathcal{S}| \geq 1$. A \emph{sampling strategy} is a triplet $\left( P_T, P_S, f \right)$, where $P_T$ is a probability distribution over $\mathcal{T} := 2^{[n]}$, $P_S$ is a probability distribution over $\mathcal{S}$ and $f$ is a real-valued function defined on the finite set\footnote{The notation $\mathcal{A}^{\ast}$ is for the free monoid over $\mathcal{A}$.}
$$
\textrm{Dom}_f := \left\{ (t, q, s)\in \mathcal{T}\times \mathcal{A}^{\ast} \times \mathcal{S}: \quad |t| = |q| \right\}.
$$
$\newline$
Remark. The number of degrees of freedom of a sampling strategy is finite and it is asymptotically equivalent to $|\mathcal{S}|^2 \, \left( 2| \mathcal{A} | + 2 \right)^n$ as $n\to\infty$.
\end{frame}

\begin{frame}
\begin{center}
\Large{Classical populations}
\end{center}
\end{frame}

\begin{frame}{What is a classical population?} 
In the context of the paper, a \emph{classical population} is a finite word $q$ over the alphabet $\mathcal{A}$. The alphabet $\mathcal{A}$ contains a distinguished element, that we will denote by $0$. We are interested in estimating how many letters in $q$ are different from $0$, i.e., the Hamming weight of $q$. In order to simplify the presentation, we will use the relative hamming weight of $q$, denoted $\omega(q)$, which is the Hamming weight of $q$ divided by its length.
\end{frame}


\begin{frame}{Example of classical population} 
The classical population could  be the population of the People's Republic of Wakanda\footnote{I am considering an idealized situation, in order to avoid all the complications of the a real-life scenario, e.g., fake data and heterogeneous properties of the population. In Wakanda everything will be as homogeneous as possible.} (each person corresponds to a position in the word $q$) and the alphabet could be $\mathcal{A} = \{0, 1\}$, where $0$ and $1$ stand for  uninfected and infected of COVID-19, respectively. In order to estimate how many people are infested with COVID-19, we will study a random sample of the population consisting of $k$ people, which will be small in comparison to the size of the population, i.e. $k$ is ``small" with respect to $n$. Finally, we will extrapolate the data from the sample to the whole population.
\end{frame}

\begin{frame}{How to estimate?} 
Given a sampling strategy $\left(P_T, P_S, f \right)$ and a classical population $q = q_1 q_2 q_3 ... q_n \in \mathcal{A}^n$. We take a random pair $(t, s) \in \mathcal{T} \times \mathcal{S}$ (using the joint distribution of $P_T$ and $P_S$). Let
$$
\tau_1 < \tau_2 < ... < \tau_{k}
$$
be the elements of $t$ written in increasing order. Let
$$
\overline{\tau }_1 < \overline{\tau}_2 < .... < \overline{\tau}_{n-k}
$$
be the elements of $[n]\backslash t$ written in increasing order. We define
$$
q_t = q_{\tau_1} q_{\tau_2} ... q_{\tau_k}, \quad q_{\overline{t}} = q_{\overline{\tau}_1} q_{\overline{\tau}_2} ... q_{\overline{\tau}_{n-k}}.
$$

By definition, we \emph{estimate} that the relative Hamming weight of $q_{\overline{t}}$ is given by $f\left(t, q_t, s\right)$.
\end{frame}

\begin{frame}{Example of estimation} 
Let's return to the example of the population of Wakanda, identified with the word $q$. We define $\mathcal{S} = \{\perp\}$, i.e., we do not use a random seed here. So, $P_S$ is trivially the uniform distribution on $\mathcal{S}$. Now, let assume that $P_T$ is the uniform distribution on subsets of size $k$ of $[n]$, i.e., 
$$P_T(t) = \frac{1}{\left(\!\!\! \begin{array}{c} n \\ k \end{array} \!\!\!\right)}$$
if $|t| = k$ and $P_T(t) = 0$ otherwise. Finally, define $$f(t, q, s) := \omega(q).$$ 

This framework is known as \emph{random sampling without replacement}.

\end{frame}

\begin{frame}{Example of estimation (continuation)} 

This sampling strategy is just a formalization of the heuristic approach assuming that the frequency of cases of COVID-19 in the whole population is the same as in our uniformly random sample. This is not necessarily the case in reality, e.g., people working in hospitals have a higher probability of getting infested than the remain of the population.
\end{frame}

\begin{frame}{Set of classical $\delta$-close states} 
Let $\delta$ be a positive real number. We define the  \emph{set of classical $\delta$-close states} as
$$
B_{t,s}^{\delta} := \left\{ q\in\mathcal{A}^n: \quad \left| \omega\left(q_{\overline{t}}\right) - f\left(t, q_t, s\right) \right| <  \delta \right\},
$$

Heuristic idea: for $\delta$, $t$ and $s$ fixed, the larger the set $B_{t,s}^{\delta}$ is, the better the function $f$ is for estimation (with respect to $t$ and $s$).
\end{frame}

\begin{frame}{Example of set of classical $\delta$-close states} 
In Wakanda example, $B_{t,s}^{\delta}$ is the set of all possibilities (after choosing $t$ and $s$) in which the frequency of infection of COVID-19 in the sample is $\delta$-close to the frequency of infection in the remaining part of the population of Wakanda.
\end{frame}

\begin{frame}{Random set of classical $\delta$-close states} 
Let $T$ and $S$ be random variables associated to the probability distributions $P_T$ and $P_S$ respectively. Notice that the pair $(T, S)$ is a random variable. Furthermore, the evaluation of $(t,s) \mapsto B_{t,s}^{\delta}$ at $(T,S)$ determines a random variable, denoted $B_{T,S}^{\delta}$ and associated to the probability distribution
$$
\textrm{Pr}\left[ B_{T,S}^{\delta} \in \Gamma \right] := \textrm{Pr}\left[ (T, S) \in \left\{(t,s)\in\mathcal{T}\times\mathcal{S} :\quad B_{t,s}^{\delta} \in \Gamma\right\} \right]
$$
where $\Gamma\subset 2^{\mathcal{A}^n}$. The call $B_{T,S}^{\delta}$ the \emph{random set of classical $\delta$-close states}.
$\newline$

Heuristic idea: for $\delta$ fixed, the ``larger" the random set $B_{T,S}^{\delta}$ is, the better the strategy is for estimation.
\end{frame}

\begin{frame}{Example of random set of classical $\delta$-close states} 
In Wakanda example, $B_{T,S}^{\delta}$ is the random set of all possibilities (before choosing $t$ and $s$) in which the frequency of infection of COVID-19 in the sample is $\delta$-close to the frequency of infection in the remaining part of the population of Wakanda.
\end{frame}

\begin{frame}{Classical $\delta$-error} 

The \emph{classical $\delta$-error} is defined as
$$
\varepsilon_c^{\delta} :=\max_{q\in\mathcal{A}^n} \textrm{Pr}\left[ B_{T,S}^{\delta} \in \left\{X\in 2^{\mathcal{A}^n}: \quad q \not\in X \right\}\right].
$$

Heuristic idea: $\varepsilon_c^{\delta}$ measures the probability that the sampling strategy fails in the worst-case scenario\footnote{Notice that the expression inside the bracket is equivalent to $q \not\in B_{T,S}^{\delta}$.}.
\end{frame}

\begin{frame}{Example of classical $\delta$-error} 
In Wakanda example, $\varepsilon_c^{\delta}$ is the largest probability, with respect to all possible populations, that the frequency of infection of the sample is not $\delta$-close to the frequency of infection of the remaining part of the population of Wakanda.
$\newline$

For $k \leq \frac{n}{2}$ (as it should be in practice), we have
$$
\varepsilon_c^{\delta} \leq 2 \exp\left( - \frac{\delta^2 k}{2} \right).
$$

Remark: As expected: a) as the size of the sample increases, the bound of the error $\varepsilon_c^{\delta}$ decrease; b) as the precision of the estimation increases, the bound of the error $\varepsilon_c^{\delta}$ increases.
\end{frame}


\begin{frame}
\begin{center}
\Large{Quantum populations}
\end{center}
\end{frame}


\begin{frame}
\begin{center}
\Large{\textbf{Presentation of Contribution 2.} }\normalsize
$\newline$
\end{center}

\begin{flushleft}
This framework is used in a new proof of the security of the \emph{quantum key distribution protocol BB84} (entanglement-based version).
\end{flushleft}
\end{frame}

\begin{frame}
\begin{center}
\Large{\textbf{Presentation of Contribution 3.} }\normalsize
$\newline$
\end{center}

\begin{flushleft}
This framework is used in a new proof of the security of the \emph{quantum oblivious-transfer from bit-commitment}.
\end{flushleft}
\end{frame}

\begin{frame}
\begin{center}
\Large{\textbf{End of my presentation} }
\end{center}
\end{frame}

\end{document}