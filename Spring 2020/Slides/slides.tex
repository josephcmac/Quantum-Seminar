\documentclass{beamer}

\usepackage[utf8]{inputenc}
\usepackage{amsmath}
\usepackage{mathtools}


%Information to be included in the title page:
\title{Presentation for the Quantum Seminar}
\author{Jos\'e Manuel Rodr\'iguez Caballero}
\institute{University of Tartu}
\date{Spring 2020}



\begin{document}

\frame{\titlepage}

\begin{frame}{Subject}
My presentation is about the paper\footnote{I will not repeat the notation from the paper in this presentation. If any listener is interested in clarification, either read the paper or ask me.}:
$\newline$

\begin{quote}
Bouman, Niek J., and Serge Fehr. "Sampling in a quantum population, and applications." Annual Cryptology Conference. Springer, Berlin, Heidelberg, 2010.
\end{quote}

My presentation will not follow the original approach of the authors in order to show that I am able to do more than merely repeat what they wrote. Also, I would like to express my personal way of looking at this subject.

\end{frame}

\begin{frame}{Contributions of the paper}
\begin{flushleft}

\textbf{Contribution 1.} The authors introduce a framework in for \textit{sampling quantum population}.
$\newline$

\textbf{Contribution 2.} This framework is used in a new proof of the security of the \emph{quantum key distribution protocol BB84} (entanglement-based version).
$\newline$

\textbf{Contribution 3.} This framework is used in a new proof of the security of the \emph{quantum oblivious-transfer from bit-commitment}.

\end{flushleft}
\end{frame}

\begin{frame}
\begin{center}
\Large{\textbf{Presentation of Contribution 1.} }\normalsize
$\newline$
\end{center}

\begin{flushleft}
The authors introduce a framework in for \textit{sampling quantum population}.
\end{flushleft}
\end{frame}

\begin{frame}{Classical sampling strategy}

Let $\mathcal{A}$ be a finite alphabet. A \emph{classical sampling strategy} is a triplet $\Psi = \left(P_T, P_S, f\right)$, where $P_T$ is a distribution over $\mathcal{T} := 2^{[n]}$, $P_S$ is a distribution over a finite set $\mathcal{S}$, $P_T$ and $P_S$ are assumed to be independent, and $f$ is a function of type\footnote{Here $\mathcal{A}^{\ast}$ is the free monoid over $\mathcal{A}$.}
\begin{eqnarray*}
& & \mathcal{T} \times  \mathcal{S} \times \mathcal{A}^{\ast}  \longrightarrow \mathbb{R}  \\
& & (t, s, q) \mapsto f_{t, s}(q)
\end{eqnarray*}
satisfying $f_{t,s}(q) = 0$ whenever $|t| \neq |q|$.
\end{frame}


\begin{frame}
\begin{center}
\Large{\textbf{Presentation of Contribution 2.} }\normalsize
$\newline$
\end{center}

\begin{flushleft}
This framework is used in a new proof of the security of the \emph{quantum key distribution protocol BB84} (entanglement-based version).
\end{flushleft}
\end{frame}

\begin{frame}
\begin{center}
\Large{\textbf{Presentation of Contribution 3.} }\normalsize
$\newline$
\end{center}

\begin{flushleft}
This framework is used in a new proof of the security of the \emph{quantum oblivious-transfer from bit-commitment}.
\end{flushleft}
\end{frame}

\begin{frame}
\begin{center}
\Large{\textbf{End of my presentation} }
\end{center}
\end{frame}

\end{document}